\documentclass{tufte-handout}
\usepackage{../braph2_dev}
%\geometry{showframe} % display margins for debugging page layout

\title{Implement a new Neural Network Data Point}

\author[The BRAPH~2 Developers]{The BRAPH~2 Developers}

\begin{document}

\maketitle

\begin{abstract}
\noindent
This is the developer tutorial for implementing a new neural network data point. 
In this Tutorial, we will explain how to create the generator file \fn{*.gen.m} for a new neural network data point, which can then be compiled by \code{braph2genesis}. All kinds of neural network data point are (direct or indirect) extensions of the base element \code{NNDataPoint}. Here, we will use as examples the neural network data point \code{NNDataPoint\_CON\_REG} (connectivity data for regression), \code{NNDataPoint\_CON\_CLA} (connectivity data for classification), \code{NNDataPoint\_Graph\_REG} (adjacency matrix for regression), \code{NNDataPoint\_Graph\_CLA} (adjacency matrix for classification), \code{NNDataPoint\_Measure\_REG} (graph measure for regression), and \code{NNDataPoint\_Measure\_CLA} (graph measure for classification).
\end{abstract}

\tableofcontents

%%%%% %%%%% %%%%% %%%%% %%%%%
\clearpage
\section{Implementation of a Data Point with Connectivity Data}

\subsection{Connectivity Data Point for Regression (\code{NNDataPoint\_CON\_REG})}

We will start by implementing in detail \code{NNDataPoint\_CON\_REG}, which is a direct extension of \code{NNDataPoint}.
A data point for regression with connectivity data \code{NNDataPoint\_CON\_REG} contains the input and target for neural network analysis with a subject with connectivity data (SubjectCON), where the input is the subject's connectivity data and the target is the subject's variables of interest.

\begin{lstlisting}[
	label=cd:m:NNDataPoint_CON_REG:header,
	caption={
		{\bf NNDataPoint\_CON\_REG element header.}
		The \code{header} section of the generator code for \fn{\_NNDataPoint\_CON\_REG.gen.m} provides the general information about the \code{NNDataPoint\_CON\_REG} element.
		}
]
%% ¡header!
NNDataPoint_CON_REG < NNDataPoint (dp, connectivity regression data point) is a data point for regression with connectivity data.
 ¥\circled{1}\circlednote{1}{defines \code{NNDataPoint\_CON\_REG} as a subclass of \code{NNDataPoint}. The moniker will be \code{dp}.}¥

%%% ¡description!
A data point for regression with connectivity data (NNDataPoint_CON_REG) 
contains the input and target for neural network analysis with a subject with connectivity data (SubjectCON).
The input is the connectivity data of the subject.
The target is obtained from the variables of interest of the subject.
\end{lstlisting}


\begin{lstlisting}[
	label={cd:m:NNDataPoint_CON_REG:prop_update},
	caption={
		{\bf NNDataPoint\_CON\_REG element prop update.}
		The \code{props\_update} section of the generator code for \fn{\_NNDataPoint\_CON\_REG.gen.m} updates the properties of the \code{NNDataPoint\_CON\_REG} element. This defines the core properties of the data point.
	}
]
%% ¡props_update!

%%% ¡prop!
NAME (constant, string) is the name of a data point for regression with connectivity data.
%%%% ¡default!
'NNDataPoint_CON_REG'

%%% ¡prop!
DESCRIPTION (constant, string) is the description of a data point for regression with connectivity data.
%%%% ¡default!
'A data point for regression with connectivity data (NNDataPoint_CON_REG) contains the input and target for neural network analysis with a subject with connectivity data (SubjectCON). The input is the connectivity data of the subject. The target is obtained from the variables of interest of the subject.'

%%% ¡prop!
TEMPLATE (parameter, item) is the template of a data point for regression with connectivity data.
%%%% ¡settings!
'NNDataPoint_CON_REG'

%%% ¡prop!
ID (data, string) is a few-letter code for a data point for regression with connectivity data.
%%%% ¡default!
'NNDataPoint_CON_REG ID'

%%% ¡prop!
LABEL (metadata, string) is an extended label of a data point for regression with connectivity data.
%%%% ¡default!
'NNDataPoint_CON_REG label'

%%% ¡prop!
NOTES (metadata, string) are some specific notes about a data point for regression with connectivity data.
%%%% ¡default!
'NNDataPoint_CON_REG notes'

%%% ¡prop!  ¥\circled{1}\circlednote{1}{The property \code{INPUT} is the input value for this data point, which is obtained directly from the connectivity data of \code{Subject\_CON} by the code under \code{¡calculate!}.}¥
INPUT (result, cell) is the input value for this data point.
%%%% ¡calculate!
value = {dp.get('SUB').get('CON')};
    
%%% ¡prop! ¥\circled{2}\circlednote{2}{The property \code{TARGET} is the target value for this data point, which is obtained directly from the variables of interest of \code{VOI\_DICT} by the code under \code{¡calculate!}.}¥
TARGET (result, cell) is the target value for this data point.
%%%% ¡calculate!
value = cellfun(@(x) dp.get('SUB').get('VOI_DICT').get('IT', x).get('V'), dp.get('TARGET_IDS'), 'UniformOutput', false);

\end{lstlisting}

\begin{lstlisting}[
	label={cd:m:NNDataPoint_CON_REG:props},
	caption={
		{\bf NNDataPoint\_CON\_REG element props.}
		The \code{props} section of generator code for \fn{\_NNDataPoint\_CON\_REG.gen.m} defines the properties to be used in \fn{NNDataPoint\_CON\_REG}.
	}
]
%% ¡props!

%%% ¡prop! ¥\circled{1}\circlednote{1}{The property \code{SUB} is a subject with connectivity data (\code{Subject\_CON}), which is used to calculated the mentioned properties \code{INPUT} and \code{TARGET}.}¥
SUB (data, item) is a subject with connectivity data.
%%%% ¡settings!
'SubjectCON'

%%% ¡prop! ¥\circled{2}\circlednote{2}{The property \code{TARGET\_IDS} defines the IDs of target, where the target IDs should be from the subject's variable-of-interest IDs.}¥
TARGET_IDS (parameter, stringlist) is a list of variable-of-interest IDs to be used as regression targets.

\end{lstlisting}

\clearpage

\begin{lstlisting}[
	label=cd:m:NNDataPoint_CON_REG:tests,
	caption={
		{\bf NNDataPoint\_CON\_REG element tests.}
		The \code{tests} section from the element generator \fn{\_NNDataPoint\_CON\_REG.gen.m}.
		A test for creating example files should be prepared to test the properties of the data point. Furthermore, additional test should be prepared for validating the value of input and target for the data point.
	}
]			
%% ¡tests!

%%% ¡excluded_props!  ¥\circled{1}\circlednote{1}{List of properties that are excluded from testing.}¥
[NNDataPoint_CON_REG.SUB]

%%% ¡test!
%%%% ¡name!
Create example files for regression  ¥\circled{2}\circlednote{2}{creates the example connectivity data files for regression analysis.}¥
%%%% ¡code!
data_dir = [fileparts(which('NNDataPoint_CON_REG')) filesep 'Example data NN REG CON XLS'];
if ~isdir(data_dir)
    mkdir(data_dir);

    % Brain Atlas  ¥\circled{3}\circlednote{3}{creates and exports the brain atlas file to the example directory.}¥
    im_ba = ImporterBrainAtlasXLS('FILE', 'desikan_atlas.xlsx');
    ba = im_ba.get('BA');
    ex_ba = ExporterBrainAtlasXLS( ...
        'BA', ba, ...
        'FILE', [data_dir filesep() 'atlas.xlsx'] ...
        );
    ex_ba.get('SAVE')
    N = ba.get('BR_DICT').get('LENGTH');

    % saves RNG
    rng_settings_ = rng(); rng('default')

    sex_options = {'Female' 'Male'};

    % Group ¥\circled{4}\circlednote{4}{creates one group of subjects with specified degree and rewiring probability configurations.}¥
    K = 2; % degree (mean node degree is 2K)
    beta = 0.3; % Rewiring probability
    gr_name = 'CON_Group_XLS';
    gr_dir = [data_dir filesep() gr_name];
    mkdir(gr_dir);
    vois = [
        {{'Subject ID'} {'Age'} {'Sex'}}
        {{} {} cell2str(sex_options)}
        ];
    for i = 1:1:100 % subject number
        sub_id = ['SubjectCON_' num2str(i)];
        % create WS graphs with random beta
        beta(i) = rand(1); ¥\circled{5}\circlednote{5}{generates random rewiring probability settings for each subject.}¥
        h = WattsStrogatz(N, K, beta(i)); % create WS graph ¥\circled{6} \twocirclednotes{6}{10}{utilize the provided degree and rewiring probability settings to generate corresponding Watts-Strogatz model graphs.}¥

        A = full(adjacency(h)); A(1:length(A)+1:numel(A)) = 0; % extract the adjacency matrix
        r = 0 + (0.5 - 0) * rand(size(A)); diffA = A - r; A(A ~= 0) = diffA(A ~= 0); % make the adjacency matrix weighted
        A = max(A, transpose(A)); % make the adjacency matrix symmetric

        writetable(array2table(A), [gr_dir filesep() sub_id '.xlsx'], 'WriteVariableNames', false) ¥\circled{7}\circlednote{7}{exports the adjacency matrix of the graph to an Excel file.}¥

        % variables of interest
        age_upperBound = 80;
        age_lowerBound = 50;
        age = age_lowerBound + beta(i)*(age_upperBound - age_lowerBound); ¥\circled{8}\circlednote{8}{associates the age value with each individual rewiring probability setting.}¥
        vois = [vois; {sub_id, age, sex_options(randi(2))}];
    end
    writetable(table(vois), [data_dir filesep() gr_name '.vois.xlsx'], 'WriteVariableNames', false) ¥\circled{9}\circlednote{9}{exports the variables of interest to an Excel file.}¥

    % reset RNG
    rng(rng_settings_)
end
%%% ¡test_functions!
function h = WattsStrogatz(N, K, beta) ¥\circled{10}¥
% H = WattsStrogatz(N,K,beta) returns a Watts-Strogatz model graph with N
% nodes, N*K edges, mean node degree 2*K, and rewiring probability beta.
%
% beta = 0 is a ring lattice, and beta = 1 is a random graph.

% Connect each node to its K next and previous neighbors. This constructs
% indices for a ring lattice.
    s = repelem((1:N)', 1, K);
    t = s + repmat(1:K, N, 1);
    t = mod(t - 1, N) + 1;
    
    % Rewire the target node of each edge with probability beta
    for source = 1:N
        switchEdge = rand(K, 1) < beta;
        
        newTargets = rand(N, 1);
        newTargets(source) = 0;
        newTargets(s(t == source)) = 0;
        newTargets(t(source, ~switchEdge)) = 0;
        
        [~, ind] = sort(newTargets, 'descend');
        t(source, switchEdge) = ind(1:nnz(switchEdge));
    end
    
    h = graph(s,t);
end

%%% ¡test! 
%%%% ¡name! ¥\circled{11}\circlednote{11}{validates the data point by using assertions to confirm that the input and target calculated values match the connectivity data and the variables of interest in the example files.}¥
Create a NNDataset containg NNDataPoint_CON_REG with simulated data
%%%% ¡code!
% Load BrainAtlas
im_ba = ImporterBrainAtlasXLS( ...
    'FILE', [fileparts(which('NNDataPoint_CON_REG')) filesep 'Example data NN REG CON XLS' filesep 'atlas.xlsx'], ...
    'WAITBAR', true ...
    );

ba = im_ba.get('BA');

% Load Group of SubjectCON
im_gr = ImporterGroupSubjectCON_XLS( ...
    'DIRECTORY', [fileparts(which('NNDataPoint_CON_REG')) filesep 'Example data NN REG CON XLS' filesep 'CON_Group_XLS'], ...
    'BA', ba, ...
    'WAITBAR', true ...
    );

gr = im_gr.get('GR');

% create an item list of NNDataPoint_CON_REG ¥\circled{12}\threecirclednotes{12}{13}{14}{creates an item list for the data points, subsequently generates the data point dictionary using the list, and then constructs the neural network dataset containing these data points.}¥
it_list = cellfun(@(x) NNDataPoint_CON_REG( ...
    'ID', x.get('ID'), ...
    'SUB', x, ...
    'TARGET_IDS', x.get('VOI_DICT').get('KEYS')), ...
    gr.get('SUB_DICT').get('IT_LIST'), ...
    'UniformOutput', false);

% create a NNDataPoint_CON_REG DICT ¥\circled{13}¥
dp_list = IndexedDictionary(...
        'IT_CLASS', 'NNDataPoint_CON_REG', ...
        'IT_LIST', it_list ...
        );

% create a NNDataset containing the NNDataPoint_CON_REG DICT ¥\circled{14}¥
d = NNDataset( ...
    'DP_CLASS', 'NNDataPoint_CON_REG', ...
    'DP_DICT', dp_list ...
    );

% Check whether the number of inputs matches ¥\circled{14}\circlednote{14}{tests the number of inputs from the dataset matches the number of subjects in the group.}¥
assert(length(d.get('INPUTS')) == gr.get('SUB_DICT').get('LENGTH'), ...
		[BRAPH2.STR ':NNDataPoint_CON_REG:' BRAPH2.FAIL_TEST], ...
		'NNDataPoint_CON_REG does not construct the dataset correctly. The number of the inputs should be the same as the number of imported subjects.' ...
		)

% Check whether the number of targets matches ¥\circled{15}\circlednote{15}{tests the number of targets from the dataset matches the number of subjects in the group.}¥
assert(length(d.get('TARGETS')) == gr.get('SUB_DICT').get('LENGTH'), ...
		[BRAPH2.STR ':NNDataPoint_CON_REG:' BRAPH2.FAIL_TEST], ...
		'NNDataPoint_CON_REG does not construct the dataset correctly. The number of the targets should be the same as the number of imported subjects.' ...
		)

% Check whether the content of input for a single data point matches ¥\circled{16}\circlednote{16}{tests the value of each input from the data point matches the subject's connectivity data.}¥
for index = 1:1:gr.get('SUB_DICT').get('LENGTH')
    individual_input = d.get('DP_DICT').get('IT', index).get('INPUT');
    known_input = {gr.get('SUB_DICT').get('IT', index).get('CON')};

    assert(isequal(individual_input, known_input), ...
        [BRAPH2.STR ':NNDataPoint_CON_REG:' BRAPH2.FAIL_TEST], ...
        'NNDataPoint_CON_REG does not construct the dataset correctly. The input value is not derived correctly.' ...
        )
end

%%% ¡test! 
%%%% ¡name!  ¥\circled{17}\circlednote{17}{executes the corresponding example scripts to ensure the functionalities.}¥
Example training-test regression
%%%% ¡code!
% ensure the example data is generated
if ~isfile([fileparts(which('NNDataPoint_CON_REG')) filesep 'Example data NN REG CON XLS' filesep 'atlas.xlsx'])
    test_NNDataPoint_CON_REG % create example files
end

example_NN_CON_REG

\end{lstlisting}

%%%%% %%%%% %%%%% %%%%% %%%%%
\clearpage
\subsection{Connectivity Data Point for Classification (\code{NNDataPoint\_CON\_CLA})}

We can now use \code{NNDataPoint\_CON\_REG} as the basis to implement the \code{NNDataPoint\_CON\_CLA}.
The parts of the code that are modified are highlighted. 

\begin{lstlisting}[
	label=cd:m:NNDataPoint_CON_CLA:header,
	caption={
		{\bf NNDataPoint\_CON\_CLA element header.}
		The \code{header} section of the generator code for \fn{\_NNDataPoint\_CON\_CLA.gen.m} provides the general information about the \code{NNDataPoint\_CON\_CLA} element.
		}
]
¤%% ¡header!¤
NNDataPoint_CON_CLA ¤< NNDataPoint¤ (dp, connectivity classification data point) ¤is a data point for¤ classification ¤with connectivity data¤.

¤%%% ¡description!¤
¤A data point for ¤classification¤ with connectivity data¤ (NNDataPoint_CON_CLA) 
¤contains the input and target for neural network analysis with a subject with connectivity data (SubjectCON).
The input is the connectivity data of the subject.
The target is obtained from the variables of interest of the subject.¤
\end{lstlisting}


\begin{lstlisting}[
	label={cd:m:NNDataPoint_CON_CLA:prop_update},
	caption={
		{\bf NNDataPoint\_CON\_CLA element prop update.}
		The \code{props\_update} section of the generator code for \fn{\_NNDataPoint\_CON\_CLA.gen.m} updates the properties of the \code{NNDataPoint\_CON\_CLA} element. This defines the core properties of the data point.
	}
]
¤%% ¡props_update!¤

¤%%% ¡prop!
NAME (constant, string) is the name of a data point for ¤classification¤ with connectivity data.
%%%% ¡default!¤
'NNDataPoint_CON_CLA'

¤%%% ¡prop!
DESCRIPTION (constant, string) is the description of a data point for ¤classification¤ with connectivity data.
%%%% ¡default!¤
'A data point for classification with connectivity data (NNDataPoint_CON_CLA) contains the input and target for neural network analysis with a subject with connectivity data (SubjectCON). The input is the connectivity data of the subject. The target is obtained from the variables of interest of the subject.'

¤%%% ¡prop!
TEMPLATE (parameter, item) is the template of a data point for ¤classification¤ with connectivity data.
%%%% ¡settings!¤
'NNDataPoint_CON_CLA'

¤%%% ¡prop!
ID (data, string) is a few-letter code for a data point for ¤classification¤ with connectivity data.
%%%% ¡default!¤
'NNDataPoint_CON_CLA ID'

¤%%% ¡prop!
LABEL (metadata, string) is an extended label of a data point for ¤classification¤ with connectivity data.
%%%% ¡default!¤
'NNDataPoint_CON_CLA label'

¤%%% ¡prop!
NOTES (metadata, string) are some specific notes about a data point for ¤classification¤ with connectivity data.
%%%% ¡default!¤
'NNDataPoint_CON_CLA notes'

¤%%% ¡prop!
INPUT (result, cell) is the input value for this data point.
%%%% ¡calculate!
value = {dp.get('SUB').get('CON')};¤
    
¤%%% ¡prop! ¥\circled{1}\circlednote{1}{defines the target value using the data point's label in the form of a string list, e.g., 'Group1'.}¥
TARGET (result, stringlist) is the target values for this data point.
%%%% ¡calculate!¤
value = dp.get('TARGET_IDS'); 

\end{lstlisting}

\begin{lstlisting}[
	label={cd:m:NNDataPoint_CON_CLA:props},
	caption={
		{\bf NNDataPoint\_CON\_CLA element props.}
		The \code{props} section of generator code for \fn{\_NNDataPoint\_CON\_CLA.gen.m} defines the properties to be used in \fn{NNDataPoint\_CON\_CLA}.
	}
]
¤%% ¡props!¤

¤%%% ¡prop!
SUB (data, item) is a subject with connectivity data.
%%%% ¡settings!
'SubjectCON'¤

¤%%% ¡prop!¤ 
¤TARGET_IDS (parameter, stringlist) is a list of variable-of-interest IDs to be used as ¤the class targets.

\end{lstlisting}

\clearpage

\begin{lstlisting}[
	label=cd:m:NNDataPoint_CON_CLA:tests,
	caption={
		{\bf NNDataPoint\_CON\_CLA element tests.}
		The \code{tests} section from the element generator \fn{\_NNDataPoint\_CON\_CLA.gen.m}.
		A test for creating example files should be prepared to test the properties of the data point. Furthermore, additional test should be prepared for validating the value of input and target for the data point.
	}
]			
¤%% ¡tests!¤

¤%%% ¡excluded_props!¤
[NNDataPoint_CON_CLA.SUB]

¤%%% ¡test!
%%%% ¡name!
Create example files
%%%% ¡code!¤
data_dir = [fileparts(which('NNDataPoint_CON_CLA')) filesep 'Example data NN CLA CON XLS'];
¤if ~isdir(data_dir)
    mkdir(data_dir);

    ...¤

    % Group 1  ¥\circled{1}\circlednote{1}{creates the first group of simulated data.}¥
    K1 = 2; % degree (mean node degree is 2K) - group 1
    beta1 = 0.3; % Rewiring probability - group 1
    gr1_name = 'CON_Group_1_XLS';
    gr1_dir = [data_dir filesep() gr1_name];
    mkdir(gr1_dir);
    vois1 = [
        {{'Subject ID'} {'Age'} {'Sex'}}
        {{} {} cell2str(sex_options)}
        ];
    for i = 1:1:50 % subject number
        sub_id = ['SubjectCON_' num2str(i)];

        h1 = WattsStrogatz(N, K1, beta1); % create two WS graph
        % figure(1) % Plot the two graphs to double-check
        % plot(h1, 'NodeColor',[1 0 0], 'EdgeColor',[0 0 0], 'EdgeAlpha',0.1, 'Layout','circle');
        % title(['Group 1: Graph with $N = $ ' num2str(N_nodes) ...
        %     ' nodes, $K = $ ' num2str(K1) ', and $\beta = $ ' num2str(beta1)], ...
        %     'Interpreter','latex')
        % axis equal

        A1 = full(adjacency(h1)); A1(1:length(A1)+1:numel(A1)) = 0; % extract the adjacency matrix
        r = 0 + (0.5 - 0)*rand(size(A1)); diffA = A1 - r; A1(A1 ~= 0) = diffA(A1 ~= 0); % make the adjacency matrix weighted
        A1 = max(A1, transpose(A1)); % make the adjacency matrix symmetric

        writetable(array2table(A1), [gr1_dir filesep() sub_id '.xlsx'], 'WriteVariableNames', false)

        % variables of interest
        vois1 = [vois1; {sub_id, randi(90), sex_options(randi(2))}];
    end
    writetable(table(vois1), [data_dir filesep() gr1_name '.vois.xlsx'], 'WriteVariableNames', false)

    % Group 2 ¥\circled{2}\twocirclednotes{2}{3}{create the second group of simulated data with different rewiring probability parameter.}¥
    K2 = 2; % degree (mean node degree is 2K) - group 2
    beta2 = 0.85; % Rewiring probability - group 2 ¥\circled{3}¥
    gr2_name = 'CON_Group_2_XLS';
    gr2_dir = [data_dir filesep() gr2_name];
    mkdir(gr2_dir);
    vois2 = [
        {{'Subject ID'} {'Age'} {'Sex'}}
        {{} {} cell2str(sex_options)}
        ];
    for i = 51:1:100
        sub_id = ['SubjectCON_' num2str(i)];

        h2 = WattsStrogatz(N, K2, beta2);
        % figure(2)
        % plot(h2, 'NodeColor',[1 0 0], 'EdgeColor',[0 0 0], 'EdgeAlpha',0.1, 'Layout','circle');
        % title(['Group 2: Graph with $N = $ ' num2str(N_nodes) ...
        %     ' nodes, $K = $ ' num2str(K2) ', and $\beta = $ ' num2str(beta2)], ...
        %     'Interpreter','latex')
        % axis equal

        A2 = full(adjacency(h2)); A2(1:length(A2)+1:numel(A2)) = 0;
        r = 0 + (0.5 - 0)*rand(size(A2)); diffA = A2 - r; A2(A2 ~= 0) = diffA(A2 ~= 0);
        A2 = max(A2, transpose(A2));

        writetable(array2table(A2), [gr2_dir filesep() sub_id '.xlsx'], 'WriteVariableNames', false)

        % variables of interest
        vois2 = [vois2; {sub_id, randi(90), sex_options(randi(2))}];
    end
    writetable(table(vois2), [data_dir filesep() gr2_name '.vois.xlsx'], 'WriteVariableNames', false)

    % Group 3 ¥\circled{4}\twocirclednotes{4}{5}{create the third group of simulated data with different rewiring probability parameter.}¥
    K3 = 2; % degree (mean node degree is 2K) - group 3
    beta3 = 0.55; % Rewiring probability - group 3  ¥\circled{5}¥
    gr3_name = 'CON_Group_3_XLS';
    gr3_dir = [data_dir filesep() gr3_name];
    mkdir(gr3_dir);
    vois3 = [
        {{'Subject ID'} {'Age'} {'Sex'}}
        {{} {} cell2str(sex_options)}
        ];
    for i = 101:1:150
        sub_id = ['SubjectCON_' num2str(i)];

        h3 = WattsStrogatz(N, K3, beta3);
        % figure(2)
        % plot(h2, 'NodeColor',[1 0 0], 'EdgeColor',[0 0 0], 'EdgeAlpha',0.1, 'Layout','circle');
        % title(['Group 2: Graph with $N = $ ' num2str(N_nodes) ...
        %     ' nodes, $K = $ ' num2str(K2) ', and $\beta = $ ' num2str(beta2)], ...
        %     'Interpreter','latex')
        % axis equal

        A3 = full(adjacency(h3)); A3(1:length(A3)+1:numel(A3)) = 0;
        r = 0 + (0.5 - 0)*rand(size(A3)); diffA = A3 - r; A3(A3 ~= 0) = diffA(A3 ~= 0);
        A3 = max(A3, transpose(A3));

        writetable(array2table(A3), [gr3_dir filesep() sub_id '.xlsx'], 'WriteVariableNames', false)

        % variables of interest
        vois3 = [vois3; {sub_id, randi(90), sex_options(randi(2))}];
    end
    writetable(table(vois3), [data_dir filesep() gr3_name '.vois.xlsx'], 'WriteVariableNames', false)

    ¤% reset RNG
    rng(rng_settings_)
end¤

¤%%% ¡test_functions!
function h = WattsStrogatz(N,K,beta)¤
...¤

¤%%% ¡test! 
%%%% ¡name!
Create a NNDataset containg NNDataPoint_CON_CLA with simulated data
¤%%%% ¡code!
% Load BrainAtlas
im_ba = ImporterBrainAtlasXLS( ...
    'FILE', [fileparts(which('NNDataPoint_CON_CLA')) filesep 'Example data NN CLA CON XLS' filesep 'atlas.xlsx'], ...
    'WAITBAR', true ...
    );

ba = im_ba.get('BA');¤

% Load Groups of SubjectCON  ¥\circled{6}\circlednote{6}{imports two groups of simulated data.}¥
im_gr1 = ImporterGroupSubjectCON_XLS( ...
    'DIRECTORY', [fileparts(which('NNDataPoint_CON_CLA')) filesep 'Example data NN CLA CON XLS' filesep 'CON_Group_1_XLS'], ...
    'BA', ba, ...
    'WAITBAR', true ...
    );

gr1 = im_gr1.get('GR');

im_gr2 = ImporterGroupSubjectCON_XLS( ...
    'DIRECTORY', [fileparts(which('NNDataPoint_CON_CLA')) filesep 'Example data NN CLA CON XLS' filesep 'CON_Group_2_XLS'], ...
    'BA', ba, ...
    'WAITBAR', true ...
    );

gr2 = im_gr2.get('GR');

% create item lists of NNDataPoint_CON_CLA  ¥\circled{7}\circlednote{7}{creates two datasets for the two groups.}¥
[~, group_folder_name] = fileparts(im_gr1.get('DIRECTORY'));
it_list1 = cellfun(@(x) NNDataPoint_CON_CLA( ...
    'ID', x.get('ID'), ...
    'SUB', x, ...
    'TARGET_IDS', {group_folder_name}), ...
    gr1.get('SUB_DICT').get('IT_LIST'), ...
    'UniformOutput', false);

[~, group_folder_name] = fileparts(im_gr2.get('DIRECTORY'));
it_list2 = cellfun(@(x) NNDataPoint_CON_CLA( ...
    'ID', x.get('ID'), ...
    'SUB', x, ...
    'TARGET_IDS', {group_folder_name}), ...
    gr2.get('SUB_DICT').get('IT_LIST'), ...
    'UniformOutput', false);

% create NNDataPoint_CON_CLA DICT items
dp_list1 = IndexedDictionary(...
        'IT_CLASS', 'NNDataPoint_CON_CLA', ...
        'IT_LIST', it_list1 ...
        );

dp_list2 = IndexedDictionary(...
        'IT_CLASS', 'NNDataPoint_CON_CLA', ...
        'IT_LIST', it_list2 ...
        );

% create a NNDataset containing the NNDataPoint_CON_CLA DICT
d1 = NNDataset( ...
    'DP_CLASS', 'NNDataPoint_CON_CLA', ...
    'DP_DICT', dp_list1 ...
    );

d2 = NNDataset( ...
    'DP_CLASS', 'NNDataPoint_CON_CLA', ...
    'DP_DICT', dp_list2 ...
    );

% Check whether the number of inputs matches ¥\circled{8}\circlednote{8}{tests the number of inputs from the dataset matches the number of subjects in the group.}¥
assert(length(d1.get('INPUTS')) == gr1.get('SUB_DICT').get('LENGTH'), ...
		[BRAPH2.STR ':NNDataPoint_CON_CLA:' BRAPH2.FAIL_TEST], ...
		'NNDataPoint_CON_CLA does not construct the dataset correctly. The number of the inputs should be the same as the number of imported subjects of group 1.' ...
		)

assert(length(d2.get('INPUTS')) == gr2.get('SUB_DICT').get('LENGTH'), ...
		[BRAPH2.STR ':NNDataPoint_CON_CLA:' BRAPH2.FAIL_TEST], ...
		'NNDataPoint_CON_CLA does not construct the dataset correctly. The number of the inputs should be the same as the number of imported subjects of group 2.' ...
		)

% Check whether the number of targets matches ¥\circled{9}\circlednote{9}{tests the number of targets from the dataset matches the number of subjects in the group.}¥
assert(length(d1.get('TARGETS')) == gr1.get('SUB_DICT').get('LENGTH'), ...
		[BRAPH2.STR ':NNDataPoint_CON_CLA:' BRAPH2.FAIL_TEST], ...
		'NNDataPoint_CON_CLA does not construct the dataset correctly. The number of the targets should be the same as the number of imported subjects of group 1.' ...
		)

assert(length(d2.get('TARGETS')) == gr2.get('SUB_DICT').get('LENGTH'), ...
		[BRAPH2.STR ':NNDataPoint_CON_CLA:' BRAPH2.FAIL_TEST], ...
		'NNDataPoint_CON_CLA does not construct the dataset correctly. The number of the targets should be the same as the number of imported subjects of group 2.' ...
		)

% Check whether the content of input for a single data point matches ¥\circled{10}\circlednote{10}{tests the value of each input from the data point matches the subject's connectivity data.}¥
for index = 1:1:gr1.get('SUB_DICT').get('LENGTH')
    individual_input = d1.get('DP_DICT').get('IT', index).get('INPUT');
    known_input = {gr1.get('SUB_DICT').get('IT', index).get('CON')};

    assert(isequal(individual_input, known_input), ...
        [BRAPH2.STR ':NNDataPoint_CON_CLA:' BRAPH2.FAIL_TEST], ...
        'NNDataPoint_CON_CLA does not construct the dataset correctly. The input value is not derived correctly.' ...
        )
end

for index = 1:1:gr2.get('SUB_DICT').get('LENGTH')
    individual_input = d2.get('DP_DICT').get('IT', index).get('INPUT');
    known_input = {gr2.get('SUB_DICT').get('IT', index).get('CON')};

    assert(isequal(individual_input, known_input), ...
        [BRAPH2.STR ':NNDataPoint_CON_CLA:' BRAPH2.FAIL_TEST], ...
        'NNDataPoint_CON_CLA does not construct the dataset correctly. The input value is not derived correctly.' ...
        )
end

¤%%% ¡test! 
%%%% ¡name!
Example training-test ¤classification¤ ¥\circled{11}\circlednote{11}{executes the corresponding example scripts to ensure the functionalities.}¥
%%%% ¡code!
% ensure the example data is generated¤
if ~isfile([fileparts(which('NNDataPoint_CON_CLA')) filesep 'Example data NN CLA CON XLS' filesep 'atlas.xlsx'])
    test_NNDataPoint_CON_CLA % create example files
¤end¤

example_NN_CON_CLA

\end{lstlisting}


%%%%% %%%%% %%%%% %%%%% %%%%%
\clearpage
\section{Implementation of a Data Point with a Graph}
\subsection{Graph Data Point for Regression (\code{NNDataPoint\_Graph\_REG})}

Now we implement \code{NNDataPoint\_Graph\_REG} based on previous codes \code{NNDataPoint\_CON\_REG}.
This neural network data point with graphs utilizes the adjacency matrix extracted from the derived graph of the subject. 
The modified parts of the code are highlighted.

\begin{lstlisting}[
	label=cd:m:NNDataPoint_Graph_REG:header,
	caption={
		{\bf NNDataPoint\_Graph\_REG element header.}
		The \code{header} section of the generator code for \fn{\_NNDataPoint\_Graph\_REG.gen.m} provides the general information about the \code{NNDataPoint\_Graph\_REG} element.
		}
]
¤%% ¡header!¤
NNDataPoint_Graph_REG ¤< NNDataPoint (dp, measure regressioni data point) is a data point for regression with¤ a graph.

¤%%% ¡description!¤
A data point for regression with a graph (NNDataPoint_Graph_REG) 
 contains both input and target for neural network analysis.
The input is the value of the adjacency matrix extracted from the derived graph of the subject.
The target is obtained from the variables of interest of the subject.
\end{lstlisting}


\begin{lstlisting}[
	label={cd:m:NNDataPoint_Graph_REG:prop_update},
	caption={
		{\bf NNDataPoint\_Graph\_REG element prop update.}
		The \code{props\_update} section of the generator code for \fn{\_NNDataPoint\_Graph\_REG.gen.m} updates the properties of the \code{NNDataPoint\_Graph\_REG} element. This defines the core properties of the data point.
	}
]
¤%% ¡props_update!¤

¤%%% ¡prop!
NAME (constant, string) is the name of a data point for regression with ¤a graph¤.
%%%% ¡default!¤
'NNDataPoint_Graph_REG'

¤%%% ¡prop!
DESCRIPTION (constant, string) is the description of a data point for regression with ¤a graph¤.
%%%% ¡default!¤
'A data point for regression with a graph (NNDataPoint_Graph_REG) contains both input and target for neural network analysis. The input is the value of the adjacency matrix extracted from the derived graph of the subject. The target is obtained from the variables of interest of the subject.'

¤%%% ¡prop!
TEMPLATE (parameter, item) is the template of a data point for regression with ¤a graph¤.
%%%% ¡settings!¤
'NNDataPoint_Graph_REG'

¤%%% ¡prop!
ID (data, string) is a few-letter code for a data point for regression with ¤a graph¤.
%%%% ¡default!¤
'NNDataPoint_Graph_REG ID'

¤%%% ¡prop!
LABEL (metadata, string) is an extended label of a data point for regression with ¤a graph¤.
%%%% ¡default!¤
'NNDataPoint_Graph_REG label'

¤%%% ¡prop!
NOTES (metadata, string) are some specific notes about a data point for regression with ¤a graph¤.
%%%% ¡default!¤
'NNDataPoint_Graph_REG notes'

¤%%% ¡prop!
INPUT (result, cell) is the input value for this data point.
%%%% ¡calculate!¤
value = dp.get('G').get('A');  ¥\circled{1}\circlednote{1}{extracts the adjacency matrix from a \code{Graph} element as the input for this data point. Note that a \code{Graph} can be any kind of \code{Graph}, including \code{GraphWU}, \code{MultigraphBUD}, and \code{MultiplexBUT}, among others.}¥
    
¤%%% ¡prop!
TARGET (result, cell) is the target value for this data point.
%%%% ¡calculate!
value = cellfun(@(x) dp.get('SUB').get('VOI_DICT').get('IT', x).get('V'), dp.get('TARGET_IDS'), 'UniformOutput', false);¤

\end{lstlisting}

\begin{lstlisting}[
	label={cd:m:NNDataPoint_Graph_REG:props},
	caption={
		{\bf NNDataPoint\_Graph\_REG element props.}
		The \code{props} section of generator code for \fn{\_NNDataPoint\_Graph\_REG.gen.m} defines the properties to be used in \fn{NNDataPoint\_Graph\_REG}.
	}
]
¤%% ¡props!¤

%%% ¡prop!  ¥\circled{1}\circlednote{1}{defines the \code{Graph} element which contains its corresponding adjacency matrix.}¥
G (data, item) is a graph.
%%%% ¡settings!
'Graph'

¤%%% ¡prop!
SUB (data, item) is a subject.
%%%% ¡settings!
'Subject'¤

¤%%% ¡prop!
TARGET_IDS (parameter, stringlist) is a list of variable-of-interest IDs to be used as the class targets.¤

\end{lstlisting}

\clearpage

\begin{lstlisting}[
	label=cd:m:NNDataPoint_Graph_REG:tests,
	caption={
		{\bf NNDataPoint\_Graph\_REG element tests.}
		The \code{tests} section from the element generator \fn{\_NNDataPoint\_Graph\_REG.gen.m}.
		A test for creating example files should be prepared to test the properties of the data point. Furthermore, additional test should be prepared for validating the value of input and target for the data point.
	}
]		
%% ¡tests!

%%% ¡excluded_props!
[NNDataPoint_Graph_REG.G NNDataPoint_Graph_REG.SUB]

%%% ¡test!
%%%% ¡name!¥\circled{1}\circlednote{1}{tests with the \code{GraphWU} element which contains weighted undirected adjacency matrix.}¥
Construct the data point with the adjacency matrix derived from its weighted undirected graph (GraphWU) 
%%%% ¡code!
¤% ensure the example data is generated
if ~isfile([fileparts(which('NNDataPoint_CON_REG')) filesep 'Example data NN REG CON XLS' filesep 'atlas.xlsx'])
    test_NNDataPoint_CON_REG % create example files
end

% Load BrainAtlas

...¤

% Analysis CON WU ¥\circled{2}\twocirclednotes{2}{3}{create the \code{AnalyzeEnsemble\_CON\_WU} element and then memorize its graph dictionary \code{G\_DICT}.}¥
a_WU = AnalyzeEnsemble_CON_WU( ...
    'GR', gr ...
    );

a_WU.memorize('G_DICT'); ¥\circled{3}¥

% create item lists of NNDataPoint_Graph_REG ¥\twocirclednotes{4}{5}{creates the \code{NNDataPoint\_Graph\_REG} element and use the \code{Graph} from \code{G\_DICT}.}¥
it_list = cellfun(@(g, sub)  ¥\circled{4}¥NNDataPoint_Graph_REG( ...
    'ID', sub.get('ID'), ...
    'G', g, ...
    'SUB', sub, ...
    'TARGET_IDS', sub.get('VOI_DICT').get('KEYS')), ...
    ¥\circled{5}¥a_WU.get('G_DICT').get('IT_LIST'), gr.get('SUB_DICT').get('IT_LIST'),...
    'UniformOutput', false);

% create NNDataPoint_Graph_REG DICT items
dp_list = IndexedDictionary(...
        'IT_CLASS', 'NNDataPoint_Graph_REG', ...
        'IT_LIST', it_list ...
        );

% create a NNDataset containing the NNDataPoint_Graph_REG DICT
d = NNDataset( ...
    'DP_CLASS', 'NNDataPoint_Graph_REG', ...
    'DP_DICT', dp_list ...
    );

% Check whether the content of input for a single data point matches ¥\circled{6}\circlednote{6}{tests whether the value of each input from the data point matches the graph's adjacency matrix.}¥
for index = 1:1:gr.get('SUB_DICT').get('LENGTH')
    individual_input = d.get('DP_DICT').get('IT', index).get('INPUT');
    known_input = a_WU.get('G_DICT').get('IT', index).get('A');

    assert(isequal(individual_input, known_input), ...
        [BRAPH2.STR ':NNDataPoint_Graph_REG:' BRAPH2.FAIL_TEST], ...
        'NNDataPoint_Graph_REG does not construct the dataset correctly. The input value is not derived correctly.' ...
        )
end

%%% ¡test!
%%%% ¡name!¥\circled{7}\circlednote{7}{tests with the \code{MultigraphBUD} element which contains the adjacency matrix of binary undirected graph at fixed densities.}¥
Construct the data point with the adjacency matrix derived from its binary undirected multigraph with fixed densities (MultigraphBUD)
%%%% ¡code!
¤% ensure the example data is generated
if ~isfile([fileparts(which('NNDataPoint_CON_REG')) filesep 'Example data NN REG CON XLS' filesep 'atlas.xlsx'])
    test_NNDataPoint_CON_REG % create example files
end

% Load BrainAtlas

...¤

% Analysis CON WU
densities = 0:25:100;

a_BUD = ¥\circled{8}¥AnalyzeEnsemble_CON_BUD( ...
    'DENSITIES', densities, ...
    'GR', gr ...
    );

a_BUD.memorize('G_DICT');

% create item lists of NNDataPoint_Graph_REG¥\threecirclednotes{8}{9}{10}{creates the \code{NNDataPoint\_Graph\_REG} element and use the \code{Graph} from \code{AnalyzeEnsemble\_CON\_BUD}.}¥
it_list = cellfun(@(g, sub) ¥\circled{9}¥ NNDataPoint_Graph_REG( ...
    'ID', sub.get('ID'), ...
    'G', g, ...
    'SUB', sub, ...
    'TARGET_IDS', sub.get('VOI_DICT').get('KEYS')), ...
    ¥\circled{10}¥a_BUD.get('G_DICT').get('IT_LIST'), gr.get('SUB_DICT').get('IT_LIST'),...
    'UniformOutput', false);

¤% create NNDataPoint_Graph_REG DICT items
dp_list = IndexedDictionary(...
        'IT_CLASS', 'NNDataPoint_Graph_REG', ...
        'IT_LIST', it_list ...
        );

...¤

%%% ¡test!
%%%% ¡name! ¥\circled{11}\circlednote{11}{tests with the \code{MultiplexWU} element which contains the adjacency matrix of weighted undirected multipex.}¥
Construct the data point with the adjacency matrix derived from its multiplex weighted undirected graph (MultiplexWU)
%%%% ¡code!
¤% ensure the example data is generated
if ~isfile([fileparts(which('SubjectCON_FUN_MP')) filesep 'Example data CON_FUN_MP XLS' filesep 'atlas.xlsx'])
    test_SubjectCON_FUN_MP % create example files
end

% Load BrainAtlas

...¤

% Analysis CON FUN MP WU
a_WU = ¥\circled{12}¥AnalyzeEnsemble_CON_FUN_MP_WU( ...
    'GR', gr ...
    );

a_WU.memorize('G_DICT');

% create item lists of NNDataPoint_Graph_REG¥\threecirclednotes{12}{13}{14}{creates the \code{NNDataPoint\_Graph\_REG} element and use the \code{Graph} from \code{AnalyzeEnsemble\_CON\_BUD}.}¥
it_list = cellfun(@(g, sub) ¥\circled{13}¥NNDataPoint_Graph_REG( ...
    'ID', sub.get('ID'), ...
    'G', g, ...
    'SUB', sub, ...
    'TARGET_IDS', sub.get('VOI_DICT').get('KEYS')), ...
    ¥\circled{14}¥a_WU.get('G_DICT').get('IT_LIST'), gr.get('SUB_DICT').get('IT_LIST'),...
    'UniformOutput', false);

¤% create NNDataPoint_Graph_REG DICT items
dp_list = IndexedDictionary(...
        'IT_CLASS', 'NNDataPoint_Graph_REG', ...
        'IT_LIST', it_list ...
        );

...¤

%%% ¡test!
%%%% ¡name!  ¥\circled{15}\circlednote{15}{tests with the \code{MultigraphBUT} element with the simulated connectivity data.}¥
Example script for binary undirected graph (MultigraphBUT) using connectivity data
%%%% ¡code!
if ~isfile([fileparts(which('NNDataPoint_CON_REG')) filesep 'Example data NN REG CON XLS' filesep 'atlas.xlsx'])
    test_NNDataPoint_CON_REG % create example files
end
example_NNCV_CON_BUT_REG

%%% ¡test!
%%%% ¡name! ¥\circled{16}\circlednote{16}{tests with the \code{MultiplexBUD} element with the simulated connectivity and functional data.}¥
Example script for binary undirected multiplex at fixed densities (MultiplexBUD) using connectivity data and functional data
%%%% ¡code!
if ~isfile([fileparts(which('NNDataPoint_CON_FUN_MP_REG')) filesep 'Example data NN REG CON_FUN_MP XLS' filesep 'atlas.xlsx'])
    test_NNDataPoint_CON_FUN_MP_REG % create example files
end
example_NNCV_CON_FUN_MP_BUD_REG

%%% ¡test!
%%%% ¡name! ¥\circled{17}\circlednote{17}{tests with the \code{MultiplexBUT} element with the simulated connectivity and functional data.}¥
Example script for binary undirected multiplex at fixed thresholds (MultiplexBUT) using connectivity data and functional data
%%%% ¡code!
if ~isfile([fileparts(which('NNDataPoint_CON_FUN_MP_REG')) filesep 'Example data NN REG CON_FUN_MP XLS' filesep 'atlas.xlsx'])
    test_NNDataPoint_CON_FUN_MP_REG % create example files
end
example_NNCV_CON_FUN_MP_BUT_REG

\end{lstlisting}

%%%%% %%%%% %%%%% %%%%% %%%%%
\clearpage
\subsection{Graph Data Point for Classification (\code{NNDataPoint\_Graph\_CLA})}

Now we implement \code{NNDataPoint\_Graph\_CLA} based on previous codes \code{NNDataPoint\_CON\_CLA}.
This neural network data point with graphs utilizes the adjacency matrix extracted from the derived graph of the subject. 
The modified parts of the code are highlighted.

\begin{lstlisting}[
	label=cd:m:NNDataPoint_Graph_CLA:header,
	caption={
		{\bf NNDataPoint\_Graph\_CLA element header.}
		The \code{header} section of the generator code for \fn{\_NNDataPoint\_Graph\_CLA.gen.m} provides the general information about the \code{NNDataPoint\_Graph\_CLA} element.
		}
]
¤%% ¡header!¤
NNDataPoint_Graph_CLA ¤< NNDataPoint¤ (dp, graph classification data point) ¤is a data point for classification with ¤a graph.

¤%%% ¡description!¤
A data point for classification with a graph (NNDataPoint_Graph_CLA) 
 contains both input and target for neural network analysis.
The input is the value of the adjacency matrix extracted from the derived graph of the subject.
The target is obtained from the variables of interest of the subject.
\end{lstlisting}


\begin{lstlisting}[
	label={cd:m:NNDataPoint_Graph_CLA:prop_update},
	caption={
		{\bf NNDataPoint\_Graph\_CLA element prop update.}
		The \code{props\_update} section of the generator code for \fn{\_NNDataPoint\_Graph\_CLA.gen.m} updates the properties of the \code{NNDataPoint\_Graph\_CLA} element. This defines the core properties of the data point.
	}
]
¤%% ¡props_update!¤

¤%%% ¡prop!
NAME (constant, string) is the name of a data point for classification with ¤a graph¤.
%%%% ¡default!¤
'NNDataPoint_Graph_CLA'

¤%%% ¡prop!
DESCRIPTION (constant, string) is the description of a data point for classification with ¤a graph¤.
%%%% ¡default!¤
'A data point for classification with a graph (NNDataPoint_Graph_CLA) contains both input and target for neural network analysis. The input is the value of the adjacency matrix extracted from the derived graph of the subject. The target is obtained from the variables of interest of the subject.'

¤%%% ¡prop!
TEMPLATE (parameter, item) is the template of a data point for classification with ¤a graph¤.
%%%% ¡settings!¤
'NNDataPoint_Graph_CLA'

¤%%% ¡prop!
ID (data, string) is a few-letter code for a data point for classification with ¤a graph¤.
%%%% ¡default!¤
'NNDataPoint_Graph_CLA ID'

¤%%% ¡prop!
LABEL (metadata, string) is an extended label of a data point for classification with ¤a graph¤.
%%%% ¡default!¤
'NNDataPoint_Graph_CLA label'

¤%%% ¡prop!
NOTES (metadata, string) are some specific notes about a data point for classification with ¤a graph¤.
%%%% ¡default!¤
'NNDataPoint_Graph_CLA notes'

¤%%% ¡prop!
INPUT (result, cell) is the input value for this data point.
%%%% ¡calculate!¤
value = dp.get('G').get('A');  ¥\circled{1}\circlednote{1}{extracts the adjacency matrix from a \code{Graph} element as the input for this data point. Note that a \code{Graph} can be any kind of \code{Graph}, including \code{GraphWU}, \code{MultigraphBUD}, and \code{MultiplexBUT}, among others.}¥
    
¤%%% ¡prop!
TARGET (result, cell) is the target value for this data point.
%%%% ¡calculate!
value = dp.get('TARGET_IDS');¤

\end{lstlisting}

\begin{lstlisting}[
	label={cd:m:NNDataPoint_Graph_CLA:props},
	caption={
		{\bf NNDataPoint\_Graph\_CLA element props.}
		The \code{props} section of generator code for \fn{\_NNDataPoint\_Graph\_CLA.gen.m} defines the properties to be used in \fn{NNDataPoint\_Graph\_CLA}.
	}
]
¤%% ¡props!¤

%%% ¡prop!  ¥\circled{1}\circlednote{1}{defines the \code{Graph} element which contains its corresponding adjacency matrix.}¥
G (data, item) is a graph.
%%%% ¡settings!
'Graph'

¤%%% ¡prop!
TARGET_IDS (parameter, stringlist) is a list of variable-of-interest IDs to be used as the class targets.¤

\end{lstlisting}

\clearpage

\begin{lstlisting}[
	label=cd:m:NNDataPoint_Graph_CLA:tests,
	caption={
		{\bf NNDataPoint\_Graph\_CLA element tests.}
		The \code{tests} section from the element generator \fn{\_NNDataPoint\_Graph\_CLA.gen.m}.
		A test for creating example files should be prepared to test the properties of the data point. Furthermore, additional test should be prepared for validating the value of input and target for the data point.
	}
]		
%% ¡tests!

%%% ¡excluded_props!
[NNDataPoint_Graph_CLA.G]

%%% ¡test!
%%%% ¡name! ¥\circled{1}\circlednote{1}{tests with the \code{GraphWU} element which contains weighted undirected adjacency matrix.}¥
Construct the data point with the adjacency matrix derived from its weighted undirected graph (GraphWU) 
%%%% ¡code!
¤% ensure the example data is generated
if ~isfile([fileparts(which('NNDataPoint_CON_CLA')) filesep 'Example data NN CLA CON XLS' filesep 'atlas.xlsx'])
    test_NNDataPoint_CON_CLA % create example files
end

% Load BrainAtlas

...¤

% Analysis CON WU
a_WU1 = ¥\circled{2}¥AnalyzeEnsemble_CON_WU( ...
    'GR', gr1 ...
    );

a_WU2 = AnalyzeEnsemble_CON_WU( ...
    'TEMPLATE', a_WU1, ...
    'GR', gr2 ...
    );

a_WU1.memorize('G_DICT');
a_WU2.memorize('G_DICT');

% create item lists of NNDataPoint_Graph_CLA ¥\threecirclednotes{2}{3}{4}{create \code{Analyzensemble\_CON\_WU} and use its \code{G\_DICT} to initialize \code{NNDataPoint\_Graph\_CLA}.}¥
[~, group_folder_name] = fileparts(im_gr1.get('DIRECTORY'));
it_list1 = cellfun(@(x) ¥\circled{3}¥NNDataPoint_Graph_CLA( ...
    'ID', x.get('ID'), ...
    'G', x, ...
    'TARGET_IDS', {group_folder_name}), ...
    ¥\circled{4}¥a_WU1.get('G_DICT').get('IT_LIST'), ...
    'UniformOutput', false);

[~, group_folder_name] = fileparts(im_gr2.get('DIRECTORY'));
it_list2 = cellfun(@(x) NNDataPoint_Graph_CLA( ...
    'ID', x.get('ID'), ...
    'G', x, ...
    'TARGET_IDS', {group_folder_name}), ...
    a_WU2.get('G_DICT').get('IT_LIST'), ...
    'UniformOutput', false);

¤% create NNDataPoint_Graph_CLA DICT items

...¤

%%% ¡test!
%%%% ¡name! ¥\circled{5}\circlednote{5}{tests with the \code{MultigraphBUD} element which contains binary undirected adjacency matrix at fixed densities.}¥
Construct the data point with the adjacency matrix derived from its binary undirected multigraph with fixed densities (MultigraphBUD)
%%%% ¡code!
¤% ensure the example data is generated
if ~isfile([fileparts(which('NNDataPoint_CON_CLA')) filesep 'Example data NN CLA CON XLS' filesep 'atlas.xlsx'])
    test_NNDataPoint_CON_CLA % create example files
end

% Load BrainAtlas

...¤

% Analysis CON WU
densities = 0:25:100;

a_BUD1 =  ¥\circled{6}¥AnalyzeEnsemble_CON_BUD( ...
    'DENSITIES', densities, ...
    'GR', gr1 ...
    );

a_BUD2 = AnalyzeEnsemble_CON_BUD( ...
    'TEMPLATE', a_BUD1, ...
    'GR', gr2 ...
    );

a_BUD1.memorize('G_DICT');
a_BUD2.memorize('G_DICT');

% create item lists of NNDataPoint_Graph_CLA¥\threecirclednotes{6}{7}{8}{create \code{Analyzensemble\_CON\_BUD} and use its \code{G\_DICT} to initialize \code{NNDataPoint\_Graph\_CLA}.}¥
[~, group_folder_name] = fileparts(im_gr1.get('DIRECTORY'));
it_list1 = cellfun(@(x) ¥\circled{7}¥NNDataPoint_Graph_CLA( ...
    'ID', x.get('ID'), ...
    'G', x, ...
    'TARGET_IDS', {group_folder_name}), ...
    ¥\circled{8}¥a_BUD1.get('G_DICT').get('IT_LIST'), ...
    'UniformOutput', false);

[~, group_folder_name] = fileparts(im_gr2.get('DIRECTORY'));
it_list2 = cellfun(@(x) NNDataPoint_Graph_CLA( ...
    'ID', x.get('ID'), ...
    'G', x, ...
    'TARGET_IDS', {group_folder_name}), ...
    a_BUD2.get('G_DICT').get('IT_LIST'), ...
    'UniformOutput', false);

% create NNDataPoint_Graph_CLA DICT items
dp_list1 = IndexedDictionary(...
        'IT_CLASS', 'NNDataPoint_Graph_CLA', ...
        'IT_LIST', it_list1 ...
        );

dp_list2 = IndexedDictionary(...
        'IT_CLASS', 'NNDataPoint_Graph_CLA', ...
        'IT_LIST', it_list2 ...
        );

¤% create a NNDataset containing the NNDataPoint_Graph_CLA DICT

...¤

%%% ¡test!
%%%% ¡name! ¥\circled{9}\circlednote{9}{tests with the \code{MultiplexWU} element which contains weighted undirected adjacency matrix from multiplex graph.}¥
Construct the data point with the adjacency matrix derived from its multiplex weighted undirected graph (MultiplexWU) 
%%%% ¡code!
¤% ensure the example data is generated
if ~isfile([fileparts(which('SubjectCON_FUN_MP')) filesep 'Example data CON_FUN_MP XLS' filesep 'atlas.xlsx'])
    test_SubjectCON_FUN_MP % create example files
end

% Load BrainAtlas

...¤

% Analysis CON FUN MP WU
a_WU1 = ¥\circled{10}¥AnalyzeEnsemble_CON_FUN_MP_WU( ...
    'GR', gr1 ...
    );

a_WU2 = AnalyzeEnsemble_CON_FUN_MP_WU( ...
    'TEMPLATE', a_WU1, ...
    'GR', gr2 ...
    );

a_WU1.memorize('G_DICT');
a_WU2.memorize('G_DICT');

% create item lists of NNDataPoint_Graph_CLA ¥\threecirclednotes{10}{11}{12}{create \code{Analyzensemble\_CON\_FUN\_MP\_WU} and use its \code{G\_DICT} to initialize \code{NNDataPoint\_Graph\_CLA}.}¥
[~, group_folder_name] = fileparts(im_gr1.get('DIRECTORY'));
it_list1 = cellfun(@(x) ¥\circled{11}¥NNDataPoint_Graph_CLA( ...
    'ID', x.get('ID'), ...
    'G', x, ...
    'TARGET_IDS', {group_folder_name}), ...
    ¥\circled{12}¥a_WU1.get('G_DICT').get('IT_LIST'), ...
    'UniformOutput', false);

[~, group_folder_name] = fileparts(im_gr2.get('DIRECTORY'));
it_list2 = cellfun(@(x) NNDataPoint_Graph_CLA( ...
    'ID', x.get('ID'), ...
    'G', x, ...
    'TARGET_IDS', {group_folder_name}), ...
    a_WU2.get('G_DICT').get('IT_LIST'), ...
    'UniformOutput', false);

¤% create NNDataPoint_Graph_CLA DICT items

...¤

%%% ¡test!
%%%% ¡name! ¥\circled{13}\circlednote{13}{tests with the \code{GraphWU} element with simulated data.}¥
Example script for weighted undirected graph (GraphWU) using connectivity data
%%%% ¡code!
if ~isfile([fileparts(which('NNDataPoint_CON_CLA')) filesep 'Example data NN CLA CON XLS' filesep 'atlas.xlsx'])
    test_NNDataPoint_CON_CLA % create example files
end
example_NNCV_CON_WU_CLA

%%% ¡test!
%%%% ¡name! ¥\circled{14}\circlednote{14}{tests with the \code{MultigraphBUD} element with simulated data.}¥
Example script for binary undirected graph at fixed densities (MultigraphBUD) using connectivity data
%%%% ¡code!
if ~isfile([fileparts(which('NNDataPoint_CON_CLA')) filesep 'Example data NN CLA CON XLS' filesep 'atlas.xlsx'])
    test_NNDataPoint_CON_CLA % create example files
end
example_NNCV_CON_BUD_CLA

%%% ¡test!
%%%% ¡name! ¥\circled{15}\circlednote{15}{tests with the \code{MultiplexWU} element with simulated data.}¥
Example script for weighted undirected multiplex (MultiplexWU) using connectivity data and functional data
%%%% ¡code!
if ~isfile([fileparts(which('NNDataPoint_CON_FUN_MP_CLA')) filesep 'Example data NN CLA CON_FUN_MP XLS' filesep 'atlas.xlsx'])
    test_NNDataPoint_CON_FUN_MP_CLA % create example files
end
example_NNCV_CON_FUN_MP_WU_CLA

\end{lstlisting}

%%%%% %%%%% %%%%% %%%%% %%%%%
\clearpage
\section{Implementation of a Data Point with Graph Measures}
\subsection{Graph Measure Data Point for Regression (\code{NNDataPoint\_Measure\_REG})}

Now we implement \code{NNDataPoint\_Measure\_REG} based on previous codes \code{NNDataPoint\_Graph\_REG}.
This neural network data point utilizes graph measures obtrained from the adjacency matrix from the derived graph of the subject. 
The modified parts of the code are highlighted.

\begin{lstlisting}[
	label=cd:m:NNDataPoint_Measure_REG:header,
	caption={
		{\bf NNDataPoint\_Measure\_REG element header.}
		The \code{header} section of the generator code for \fn{\_NNDataPoint\_Measure\_REG.gen.m} provides the general information about the \code{NNDataPoint\_Measure\_REG} element.
		}
]
¤%% ¡header!¤
NNDataPoint_Measure_REG ¤< NNDataPoint¤ (dp, measure regression data point) ¤is a data point for regression with¤ graph measures.

¤%%% ¡description!¤
A data point for regression with graph measures (NNDataPoint_Measure_REG) 
 contains both input and target for neural network analysis.
The input is the value of the graph measures (e.g. Degree, DegreeAv, and Distance), 
 calculated from the derived graph of the subject.
The target is obtained from the variables of interest of the subject.
\end{lstlisting}


\begin{lstlisting}[
	label={cd:m:NNDataPoint_Measure_REG:prop_update},
	caption={
		{\bf NNDataPoint\_Measure\_REG element prop update.}
		The \code{props\_update} section of the generator code for \fn{\_NNDataPoint\_Measure\_REG.gen.m} updates the properties of the \code{NNDataPoint\_Measure\_REG} element. This defines the core properties of the data point.
	}
]
¤%% ¡props_update!¤

¤%%% ¡prop!
NAME (constant, string) is the name of a data point for regression with ¤graph measures¤.
%%%% ¡default!¤
'NNDataPoint_Measure_REG'

¤%%% ¡prop!
DESCRIPTION (constant, string) is the description of a data point for regression with ¤graph measures¤.
%%%% ¡default!¤
'A data point for regression with graph measures (NNDataPoint_Measure_REG) contains both input and target for neural network analysis. The input is the value of the graph measures (e.g. Degree, DegreeAv, and Distance), calculated from the derived graph of the subject. The target is obtained from the variables of interest of the subject.'

¤%%% ¡prop!
TEMPLATE (parameter, item) is the template of a data point for regression with ¤graph measures¤.
%%%% ¡settings!¤
'NNDataPoint_Measure_REG'

¤%%% ¡prop!
ID (data, string) is a few-letter code for a data point for regression with ¤graph measures¤.
%%%% ¡default!¤
'NNDataPoint_Measure_REG ID'

¤%%% ¡prop!
LABEL (metadata, string) is an extended label of a data point for regression with ¤graph measures¤.
%%%% ¡default!¤
'NNDataPoint_Measure_REG label'

¤%%% ¡prop!
NOTES (metadata, string) are some specific notes about a data point for regression with ¤graph measures¤.
%%%% ¡default!¤
'NNDataPoint_Measure_REG notes'

¤%%% ¡prop!
INPUT (result, cell) is the input value for this data point.
%%%% ¡calculate!¤ 
value = cellfun(@(m_class) dp.get('G').get('MEASURE', m_class).get('M'), dp.get('M_LIST'), 'UniformOutput', false);  ¥\circled{1}\circlednote{1}{calculates the graph measures, specified with \code{M\_LIST}, from a \code{Graph} element for this data point. Note that a \code{Graph} can be any kind of \code{Graph}, including \code{GraphWU}, \code{MultigraphBUD}, and \code{MultiplexBUT}, among others.}¥
    
¤%%% ¡prop!
TARGET (result, cell) is the target value for this data point.
%%%% ¡calculate!
value = cellfun(@(x) dp.get('SUB').get('VOI_DICT').get('IT', x).get('V'), dp.get('TARGET_IDS'), 'UniformOutput', false);¤

\end{lstlisting}

\begin{lstlisting}[
	label={cd:m:NNDataPoint_Measure_REG:props},
	caption={
		{\bf NNDataPoint\_Measure\_REG element props.}
		The \code{props} section of generator code for \fn{\_NNDataPoint\_Measure\_REG.gen.m} defines the properties to be used in \fn{NNDataPoint\_Measure\_REG}.
	}
]
¤%% ¡props!¤

¤%%% ¡prop!  
G (data, item) is a graph.
%%%% ¡settings!
'Graph'¤

%%% ¡prop! ¥\circled{1}\circlednote{1}{defines the graph measure list which will be obtained as \code{INPUT} for this data point.}¥
M_LIST (parameter, classlist) is a list of graph measure to be used as the input

¤%%% ¡prop!
SUB (data, item) is a subject.
%%%% ¡settings!
'Subject'¤

¤%%% ¡prop!
TARGET_IDS (parameter, stringlist) is a list of variable-of-interest IDs to be used as the class targets.¤

\end{lstlisting}

\clearpage

\begin{lstlisting}[
	label=cd:m:NNDataPoint_Measure_REG:tests,
	caption={
		{\bf NNDataPoint\_Measure\_REG element tests.}
		The \code{tests} section from the element generator \fn{\_NNDataPoint\_Measure\_REG.gen.m}.
		A test for creating example files should be prepared to test the properties of the data point. Furthermore, additional test should be prepared for validating the value of input and target for the data point.
	}
]		
%% ¡tests!

%%% ¡excluded_props!
[NNDataPoint_Measure_REG.G NNDataPoint_Measure_REG.SUB]

%%% ¡test!
%%%% ¡name! 
Construct the data point with the adjacency matrix derived from its weighted undirected graph (GraphWU) 
%%%% ¡code!
¤% ensure the example data is generated
if ~isfile([fileparts(which('NNDataPoint_CON_REG')) filesep 'Example data NN REG CON XLS' filesep 'atlas.xlsx'])
    test_NNDataPoint_CON_REG % create example files
end

% Load BrainAtlas

...¤

% Analysis CON WU
a_WU = ¥\circled{1}¥AnalyzeEnsemble_CON_WU( ...
    'GR', gr ...
    );

a_WU.get('MEASUREENSEMBLE', 'Degree').get('M');  ¥\circled{2}¥
a_WU.get('MEASUREENSEMBLE', 'DegreeAv').get('M'); ¥\circled{3}¥
a_WU.get('MEASUREENSEMBLE', 'Distance').get('M'); ¥\circled{4}¥
¥\fourcirclednotes{1}{2}{3}{4}{create a \code{AnalyzeEnsemble\_CON\_WU} and add various kinds of graph measure with the \code{GraphWU} element which contains weighted undirected adjacency matrix.}¥
% create item lists of NNDataPoint_Measure_REG
it_list = cellfun(@(g, sub)  ¥\circled{5}¥NNDataPoint_Measure_REG( ...
    'ID', sub.get('ID'), ...
    'G', g, ...
    'M_LIST', {'Degree' 'DegreeAv' 'Distance'}, ...
    'SUB', sub, ...
    'TARGET_IDS', sub.get('VOI_DICT').get('KEYS')), ...
     ¥\circled{6}¥a_WU.get('G_DICT').get('IT_LIST'), gr.get('SUB_DICT').get('IT_LIST'),...
    'UniformOutput', false);
¥\twocirclednotes{5}{6}{use \code{AnalyzeEnsemble\_CON\_WU}'s \code{G\_DICT} to set up a \code{NNDataPoint\_Measure\_REG}.}¥
¤% create NNDataPoint_Measure_REG DICT items
dp_list = IndexedDictionary(...
        'IT_CLASS', 'NNDataPoint_Measure_REG', ...
        'IT_LIST', it_list ...
        );

...¤

%%% ¡test!
%%%% ¡name! ¥\circled{7}\fourcirclednotes{7}{8}{9}{10}{tests various kinds of graph measure with the \code{MultigraphBUD}.}¥
Construct the data point with the adjacency matrix derived from its binary undirected multigraph with fixed densities (MultigraphBUD)
¤%%%% ¡code!
% ensure the example data is generated
if ~isfile([fileparts(which('NNDataPoint_CON_REG')) filesep 'Example data NN REG CON XLS' filesep 'atlas.xlsx'])
    test_NNDataPoint_CON_REG % create example files
end

% Load BrainAtlas

...¤

% Analysis CON WU
densities = 0:25:100;

a_BUD = ¥\circled{8}¥AnalyzeEnsemble_CON_BUD( ...
    'DENSITIES', densities, ...
    'GR', gr ...
    );

a_BUD.get('MEASUREENSEMBLE', 'Degree').get('M');
a_BUD.get('MEASUREENSEMBLE', 'DegreeAv').get('M');
a_BUD.get('MEASUREENSEMBLE', 'Distance').get('M');

% create item lists of NNDataPoint_Measure_REG
it_list = cellfun(@(g, sub) ¥\circled{9}¥NNDataPoint_Measure_REG( ...
    'ID', sub.get('ID'), ...
    'G', g, ...
    'M_LIST', {'Degree' 'DegreeAv' 'Distance'}, ...
    'SUB', sub, ...
    'TARGET_IDS', sub.get('VOI_DICT').get('KEYS')), ...
    ¥\circled{10}¥a_BUD.get('G_DICT').get('IT_LIST'), gr.get('SUB_DICT').get('IT_LIST'),...
    'UniformOutput', false);

¤% create NNDataPoint_CON_CLA DICT items
dp_list = IndexedDictionary(...
        'IT_CLASS', 'NNDataPoint_Measure_REG', ...
        'IT_LIST', it_list ...
        );

...¤

%%% ¡test!
%%%% ¡name! ¥\circled{11}\fourcirclednotes{11}{12}{13}{14}{tests various kinds of graph measure with the \code{MultiplexWU}.}¥
Construct the data point with the adjacency matrix derived from its multiplex weighted undirected graph (MultiplexWU) 
¤%%%% ¡code!
% ensure the example data is generated
if ~isfile([fileparts(which('SubjectCON_FUN_MP')) filesep 'Example data CON_FUN_MP XLS' filesep 'atlas.xlsx'])
    test_SubjectCON_FUN_MP % create example files
end

% Load BrainAtlas

...¤

% Analysis CON FUN MP WU
a_WU = ¥\circled{12}¥AnalyzeEnsemble_CON_FUN_MP_WU( ...
    'GR', gr ...
    );

% To be added the multiplex measures
a_WU.get('MEASUREENSEMBLE', 'Degree').get('M');
a_WU.get('MEASUREENSEMBLE', 'DegreeAv').get('M');
a_WU.get('MEASUREENSEMBLE', 'Distance').get('M');

% create item lists of NNDataPoint_Measure_REG
it_list = cellfun(@(g, sub) ¥\circled{13}¥NNDataPoint_Measure_REG( ...
    'ID', sub.get('ID'), ...
    'G', g, ...
    'M_LIST', {'Degree' 'DegreeAv' 'Distance'}, ...
    'SUB', sub, ...
    'TARGET_IDS', sub.get('VOI_DICT').get('KEYS')), ...
    ¥\circled{14}¥a_WU.get('G_DICT').get('IT_LIST'), gr.get('SUB_DICT').get('IT_LIST'),...
    'UniformOutput', false);

¤% create NNDataPoint_Measure_REG DICT items

...¤

%%% ¡test!
%%%% ¡name! ¥\circled{15}\circlednote{15}{tests various kinds of graph measure with the \code{GraphWU} using example data.}¥
Example script for weighted undirected graph (GraphWU) using connectivity data
%%%% ¡code!
if ~isfile([fileparts(which('NNDataPoint_CON_REG')) filesep 'Example data NN REG CON XLS' filesep 'atlas.xlsx'])
    test_NNDataPoint_CON_REG % create example files
end
example_NNCV_CON_WU_M_REG

%%% ¡test!
%%%% ¡name! ¥\circled{16}\circlednote{16}{tests various kinds of graph measure with the \code{MultiplexWU} using example data.}¥
Example script for weighted undirected multiplex (MultiplexWU) using connectivity data and functional data
%%%% ¡code!
if ~isfile([fileparts(which('NNDataPoint_CON_FUN_MP_REG')) filesep 'Example data NN REG CON_FUN_MP XLS' filesep 'atlas.xlsx'])
    test_NNDataPoint_CON_FUN_MP_REG % create example files
end
example_NNCV_CON_FUN_MP_WU_M_REG

\end{lstlisting}

%%%%% %%%%% %%%%% %%%%% %%%%%
\clearpage
\subsection{Graph Measure Data Point for Classification (\code{NNDataPoint\_Measure\_CLA})}

Now we implement \code{NNDataPoint\_Measure\_CLA} based on previous codes \code{NNDataPoint\_Graph\_CLA}.
This neural network data point utilizes graph measures obtrained from the adjacency matrix from the derived graph of the subject. 
The modified parts of the code are highlighted.

\begin{lstlisting}[
	label=cd:m:NNDataPoint_Measure_CLA:header,
	caption={
		{\bf NNDataPoint\_Measure\_CLA element header.}
		The \code{header} section of the generator code for \fn{\_NNDataPoint\_Measure\_CLA.gen.m} provides the general information about the \code{NNDataPoint\_Measure\_CLA} element.
		}
]
¤%% ¡header!¤
NNDataPoint_Measure_CLA ¤< NNDataPoint¤ (dp, measure classification data point) ¤is a data point for classification with¤ graph measures.

¤%%% ¡description!¤
A data point for classification with graph measures (NNDataPoint_Measure_CLA) 
 contains both input and target for neural network analysis.
The input is the value of the graph measures (e.g. Degree, DegreeAv, and Distance), 
 calculated from the derived graph of the subject.
The target is obtained from the variables of interest of the subject.
\end{lstlisting}

\begin{lstlisting}[
	label={cd:m:NNDataPoint_Measure_CLA:prop_update},
	caption={
		{\bf NNDataPoint\_Measure\_CLA element prop update.}
		The \code{props\_update} section of the generator code for \fn{\_NNDataPoint\_Measure\_CLA.gen.m} updates the properties of the \code{NNDataPoint\_Measure\_CLA} element. This defines the core properties of the data point.
	}
]
¤%% ¡props_update!¤

¤%%% ¡prop!
NAME (constant, string) is the name of a data point for classification with ¤graph measures¤.
%%%% ¡default!¤
'NNDataPoint_Measure_CLA'

¤%%% ¡prop!
DESCRIPTION (constant, string) is the description of a data point for classification with ¤graph measures¤.
%%%% ¡default!¤
'A data point for classification with graph measures (NNDataPoint_Measure_CLA) contains both input and target for neural network analysis. The input is the value of the graph measures (e.g. Degree, DegreeAv, and Distance), calculated from the derived graph of the subject. The target is obtained from the variables of interest of the subject.'

¤%%% ¡prop!
TEMPLATE (parameter, item) is the template of a data point for classification with ¤graph measures¤.
%%%% ¡settings!¤
'NNDataPoint_Measure_CLA'

¤%%% ¡prop!
ID (data, string) is a few-letter code for a data point for classification with ¤graph measures¤.
%%%% ¡default!¤
'NNDataPoint_Measure_CLA ID'

¤%%% ¡prop!
LABEL (metadata, string) is an extended label of a data point for classification with ¤graph measures¤.
%%%% ¡default!¤
'NNDataPoint_Measure_CLA label'

¤%%% ¡prop!
NOTES (metadata, string) are some specific notes about a data point for classification with ¤graph measures¤.
%%%% ¡default!¤
'NNDataPoint_Measure_CLA notes'

¤%%% ¡prop!
INPUT (result, cell) is the input value for this data point.
%%%% ¡calculate!¤ 
value = cellfun(@(m_class) dp.get('G').get('MEASURE', m_class).get('M'), dp.get('M_LIST'), 'UniformOutput', false);  ¥\circled{1}\circlednote{1}{calculates or extract the graph measures, which are specified with \code{M\_LIST} from a \code{Graph} element for this data point. Note that a \code{Graph} can be any kind of \code{Graph}, including \code{GraphWU}, \code{MultigraphBUD}, and \code{MultiplexBUT}, among others.}¥
    
¤%%% ¡prop!
TARGET (result, cell) is the target value for this data point.
%%%% ¡calculate!
value = dp.get('TARGET_IDS');¤

\end{lstlisting}

\begin{lstlisting}[
	label={cd:m:NNDataPoint_Measure_CLA:props},
	caption={
		{\bf NNDataPoint\_Measure\_CLA element props.}
		The \code{props} section of generator code for \fn{\_NNDataPoint\_Measure\_CLA.gen.m} defines the properties to be used in \fn{NNDataPoint\_Measure\_CLA}.
	}
]
¤%% ¡props!¤

¤%%% ¡prop!  
G (data, item) is a graph.
%%%% ¡settings!
'Graph'¤

%%% ¡prop! ¥\circled{1}\circlednote{1}{defines the graph measure list which will be obtained as \code{INPUT} for this data point.}¥
M_LIST (parameter, classlist) is a list of graph measure to be used as the input

¤%%% ¡prop!
TARGET_IDS (parameter, stringlist) is a list of variable-of-interest IDs to be used as the class targets.¤

\end{lstlisting}

\clearpage

\begin{lstlisting}[
	label=cd:m:NNDataPoint_Measure_CLA:tests,
	caption={
		{\bf NNDataPoint\_Measure\_CLA element tests.}
		The \code{tests} section from the element generator \fn{\_NNDataPoint\_Measure\_CLA.gen.m}.
		A test for creating example files should be prepared to test the properties of the data point. Furthermore, additional test should be prepared for validating the value of input and target for the data point.
	}
]		
%% ¡tests!

%%% ¡excluded_props!
[NNDataPoint_Measure_CLA.G]

%%% ¡test!
%%%% ¡name! 
Construct the data point with the graph measures derived from its weighted undirected graph (GraphWU) 
%%%% ¡code!
¤% ensure the example data is generated
if ~isfile([fileparts(which('NNDataPoint_CON_CLA')) filesep 'Example data NN CLA CON XLS' filesep 'atlas.xlsx'])
    test_NNDataPoint_CON_CLA % create example files
end

% Load BrainAtlas

...¤

% Analysis CON WU
a_WU1 = ¥\circled{1}¥AnalyzeEnsemble_CON_WU( ...
    'GR', gr1 ...
    );

a_WU2 = AnalyzeEnsemble_CON_WU( ...
    'TEMPLATE', a_WU1, ...
    'GR', gr2 ...
    );

a_WU1.get('MEASUREENSEMBLE', 'Degree').get('M');  ¥\circled{2}¥
a_WU1.get('MEASUREENSEMBLE', 'DegreeAv').get('M'); ¥\circled{3}¥ 
a_WU1.get('MEASUREENSEMBLE', 'Distance').get('M'); ¥\circled{4}¥
¥\fourcirclednotes{1}{2}{3}{4}{create a \code{AnalyzeEnsemble\_CON\_WU} and add various kinds of graph measure with the \code{GraphWU} element which contains weighted undirected adjacency matrix.}¥
a_WU2.get('MEASUREENSEMBLE', 'Degree').get('M');
a_WU2.get('MEASUREENSEMBLE', 'DegreeAv').get('M');
a_WU2.get('MEASUREENSEMBLE', 'Distance').get('M');
¥\twocirclednotes{5}{6}{use \code{AnalyzeEnsemble\_CON\_WU}'s \code{G\_DICT} to set up a \code{NNDataPoint\_Measure\_CLA}.}¥
% create item lists of NNDataPoint_Measure_CLA
[~, group_folder_name] = fileparts(im_gr1.get('DIRECTORY'));
it_list1 = cellfun(@(x) ¥\circled{5}¥NNDataPoint_Measure_CLA( ...
    'ID', x.get('ID'), ...
    'G', x, ...
    'M_LIST', {'Degree' 'DegreeAv' 'Distance'}, ...
    'TARGET_IDS', {group_folder_name}), ...
    ¥\circled{6}¥a_WU1.get('G_DICT').get('IT_LIST'), ...
    'UniformOutput', false);

¤% create NNDataPoint_Measure_CLA DICT items

...¤

%%% ¡test!
%%%% ¡name!  ¥\circled{7}\fourcirclednotes{7}{8}{9}{10}{tests various kinds of graph measure with the \code{MultigraphBUD}.}¥
Construct the data point with the graph measures derived from its binary undirected multigraph with fixed densities (MultigraphBUD)
%%%% ¡code!
¤% ensure the example data is generated
if ~isfile([fileparts(which('NNDataPoint_CON_CLA')) filesep 'Example data NN CLA CON XLS' filesep 'atlas.xlsx'])
    test_NNDataPoint_CON_CLA % create example files
end

% Load BrainAtlas

...¤

% Analysis CON WU
densities = 0:25:100;

a_BUD1 = ¥\circled{8}¥AnalyzeEnsemble_CON_BUD( ...
    'DENSITIES', densities, ...
    'GR', gr1 ...
    );

a_BUD2 = AnalyzeEnsemble_CON_BUD( ...
    'TEMPLATE', a_BUD1, ...
    'GR', gr2 ...
    );

a_BUD1.get('MEASUREENSEMBLE', 'Degree').get('M'); ¥\circled{9}¥
a_BUD1.get('MEASUREENSEMBLE', 'DegreeAv').get('M');
a_BUD1.get('MEASUREENSEMBLE', 'Distance').get('M');

a_BUD2.get('MEASUREENSEMBLE', 'Degree').get('M');
a_BUD2.get('MEASUREENSEMBLE', 'DegreeAv').get('M');
a_BUD2.get('MEASUREENSEMBLE', 'Distance').get('M');

% create item lists of NNDataPoint_Measure_CLA
[~, group_folder_name] = fileparts(im_gr1.get('DIRECTORY'));
it_list1 = cellfun(@(x) ¥\circled{10}¥NNDataPoint_Measure_CLA( ...
    'ID', x.get('ID'), ...
    'G', x, ...
    'M_LIST', {'Degree' 'DegreeAv' 'Distance'}, ...
    'TARGET_IDS', {group_folder_name}), ...
    a_BUD1.get('G_DICT').get('IT_LIST'), ...
    'UniformOutput', false);

[~, group_folder_name] = fileparts(im_gr2.get('DIRECTORY'));
it_list2 = cellfun(@(x) NNDataPoint_Measure_CLA( ...
    'ID', x.get('ID'), ...
    'G', x, ...
    'M_LIST', {'Degree' 'DegreeAv' 'Distance'}, ...
    'TARGET_IDS', {group_folder_name}), ...
    a_BUD2.get('G_DICT').get('IT_LIST'), ...
    'UniformOutput', false);

¤% create NNDataPoint_Measure_CLA items
dp_list1 = IndexedDictionary(...
        'IT_CLASS', 'NNDataPoint_Measure_CLA', ...
        'IT_LIST', it_list1 ...
        );

...¤

%%% ¡test!
%%%% ¡name!  ¥\circled{11}\fourcirclednotes{11}{12}{13}{14}{tests various kinds of graph measure with the \code{MultigraphBUD}.}¥
Construct the data point with the graph measures derived from its multiplex weighted undirected graph (MultiplexWU) 
%%%% ¡code!
¤% ensure the example data is generated
if ~isfile([fileparts(which('SubjectCON_FUN_MP')) filesep 'Example data CON_FUN_MP XLS' filesep 'atlas.xlsx'])
    test_SubjectCON_FUN_MP % create example files
end

% Load BrainAtlas

...¤

% Analysis CON FUN MP WU
a_WU1 =  ¥\circled{12}¥AnalyzeEnsemble_CON_FUN_MP_WU( ...
    'GR', gr1 ...
    );

a_WU2 = AnalyzeEnsemble_CON_FUN_MP_WU( ...
    'TEMPLATE', a_WU1, ...
    'GR', gr2 ...
    );

% To be added the multiplex measures
a_WU1.get('MEASUREENSEMBLE', 'Degree').get('M'); ¥\circled{13}¥
a_WU1.get('MEASUREENSEMBLE', 'DegreeAv').get('M');
a_WU1.get('MEASUREENSEMBLE', 'Distance').get('M');

a_WU2.get('MEASUREENSEMBLE', 'Degree').get('M');
a_WU2.get('MEASUREENSEMBLE', 'DegreeAv').get('M');
a_WU2.get('MEASUREENSEMBLE', 'Distance').get('M');

% create item lists of NNDataPoint_Graph_CLA
[~, group_folder_name] = fileparts(im_gr1.get('DIRECTORY'));
it_list1 = cellfun(@(x) ¥\circled{14}¥NNDataPoint_Measure_CLA( ...
    'ID', x.get('ID'), ...
    'G', x, ...
    'M_LIST', {'Degree' 'DegreeAv' 'Distance'}, ...
    'TARGET_IDS', {group_folder_name}), ...
    a_WU1.get('G_DICT').get('IT_LIST'), ...
    'UniformOutput', false);

[~, group_folder_name] = fileparts(im_gr2.get('DIRECTORY'));
it_list2 = cellfun(@(x) NNDataPoint_Measure_CLA( ...
    'ID', x.get('ID'), ...
    'G', x, ...
    'M_LIST', {'Degree' 'DegreeAv' 'Distance'}, ...
    'TARGET_IDS', {group_folder_name}), ...
    a_WU2.get('G_DICT').get('IT_LIST'), ...
    'UniformOutput', false);

¤% create NNDataPoint_Measure_CLA DICT items
dp_list1 = IndexedDictionary(...
        'IT_CLASS', 'NNDataPoint_Measure_CLA', ...
        'IT_LIST', it_list1 ...
        );

...¤

%%% ¡test!
%%%% ¡name! ¥\circled{15}\circlednote{15}{tests various kinds of graph measure with the \code{MultigraphBUD} using example connectivity data.}¥
Example script for binary undirected graph at fixed densities (GraphBUD) using connectivity data
%%%% ¡code!
if ~isfile([fileparts(which('NNDataPoint_CON_CLA')) filesep 'Example data NN CLA CON XLS' filesep 'atlas.xlsx'])
    test_NNDataPoint_CON_CLA % create example files
end
example_NNCV_CON_BUD_M_CLA

%%% ¡test!
%%%% ¡name! ¥\circled{16}\circlednote{16}{tests various kinds of graph measure with the \code{MultigraphBUT} using example data.}¥
Example script for binary undirected graph at fixed thresholds (MultigraphBUT) using connectivity data
%%%% ¡code!
if ~isfile([fileparts(which('NNDataPoint_CON_CLA')) filesep 'Example data NN CLA CON XLS' filesep 'atlas.xlsx'])
    test_NNDataPoint_CON_CLA % create example files
end
example_NNCV_CON_BUT_M_CLA

%%% ¡test!
%%%% ¡name! ¥\circled{17}\circlednote{17}{tests various kinds of graph measure with the \code{MultigraphBUD} using example connectivity data.}¥
Example script for binary undirected graph at fixed densities (MultigraphBUD) using connectivity data
%%%% ¡code!
if ~isfile([fileparts(which('NNDataPoint_CON_CLA')) filesep 'Example data NN CLA CON XLS' filesep 'atlas.xlsx'])
    test_NNDataPoint_CON_CLA % create example files
end
example_NNCV_CON_BUD_M_CLA

%%% ¡test!
%%%% ¡name! ¥\circled{18}\circlednote{18}{tests various kinds of graph measure with the \code{MultiplexBUD} using example functional data.}¥
Example script for binary undirected multiplex at fixed densities (MultiplexBUD) using connectivity data and functional data
%%%% ¡code!
if ~isfile([fileparts(which('NNDataPoint_CON_FUN_MP_CLA')) filesep 'Example data NN CLA CON_FUN_MP XLS' filesep 'atlas.xlsx'])
    test_NNDataPoint_CON_FUN_MP_CLA % create example files
end
example_NNCV_CON_FUN_MP_BUD_M_CLA

%%% ¡test!
%%%% ¡name! ¥\circled{19}\circlednote{19}{tests various kinds of graph measure with the \code{MultiplexBUT} using example functional data.}¥
Example script for binary undirected multiplex at fixed thresholds (MultiplexBUT) using connectivity data and functional data
%%%% ¡code!
if ~isfile([fileparts(which('NNDataPoint_CON_FUN_MP_CLA')) filesep 'Example data NN CLA CON_FUN_MP XLS' filesep 'atlas.xlsx'])
    test_NNDataPoint_CON_FUN_MP_CLA % create example files
end
example_NNCV_CON_FUN_MP_BUT_M_CLA

\end{lstlisting}

%\bibliography{biblio}
%\bibliographystyle{plainnat}

\end{document}